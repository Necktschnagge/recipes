\documentclass[a4paper]{book}
\usepackage[textwidth=17cm, textheight=25cm]{geometry}

\input{../latex-std/lang-de.tex}

\usepackage{enumerate}
\usepackage{textcomp}

\title{Rezeptsammlung von Informatikern}
\date{Letzte Änderungen: \today}
\author{Maximilian Starke}

\begin{document}

\maketitle
\tableofcontents

\chapter*{Einleitung}

In diesem Buch sammeln wir Rezepte zum Kochen und Backen.
Jeder darf Rezepte einreichen. 

\chapter{Backrezepte}

\section{Rezepte mit Hefeteig}

\newpage
\subsection{Plätzchen}

\subsubsection{bereitgestellt von}
	Maximilian Starke
\subsubsection{Ursprung}
	Torgau in Sachsen (Deutschland)
\subsubsection{Zutaten}
\begin{center}
	\begin{tabular}{|l|rl|r|}
		\hline
		\textbf{Zutat} & \textbf{Menge} & \textbf{Maßeinheit} & \textbf{Bemerkungen}\\
		\hline
		Mehl & 400 & g & \\
		Butter & 300 & g & \\
		\hline
		Mandeln (blanchiert und gemahlen) & 50 & g & \\
		Backpulver & $\frac{1}{2}$ & Päckchen & \\
		\hline
		Zucker & 120 & g & \\
		Vanillinzucker & 4 & Teelöffel & \\
		\hline
		Ei & 1 &  Stück & \\
		\hline
	\end{tabular}
\end{center}

\subsubsection{Zubereitung}

\begin{enumerate}[(1)]
	\item Man verrühre alle Zutaten gründlich.
	\item Man positioniere Backpapier auf einem Backblech.
	\item Man bringe die Kekse in die gewünschte Form und breite sie auf dem Backpapier aus.
\end{enumerate}
\subsubsection{Backzeit}
	Man backe die Kekse bei Umluft und einer Temperatur von 160{\textcelsius} ca. 10 Minuten lang.
\end{document}