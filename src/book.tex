\documentclass[a4paper]{book}
\usepackage[textwidth=17cm, textheight=25cm]{geometry}

\input{../latex-std/lang-de.tex}

\usepackage{enumerate} % \begin{enumerate} ... \end{enumerate}
\usepackage{textcomp} % \textcelsius (°C) 

\title{Rezeptsammlung von Informatikern}
\date{Letzte Änderungen: \today}
\author{Maximilian Starke}

\begin{document}

\maketitle
\tableofcontents

\chapter*{Einleitung}

In diesem Buch sammeln wir Rezepte zum Kochen und Backen.
Jeder darf Rezepte einreichen. 

\chapter{Backrezepte}

\section{Rezepte mit Quark}

\newpage
\subsection{Quark-Torte}

\subsubsection{bereitgestellt von}
	Maximilian Starke
\subsubsection{Ursprung}
	Torgau in Sachsen (Deutschland)
\subsubsection{Zutaten}
\begin{center}
	\begin{tabular}{|l|rl|}
		\hline
		\textbf{Zutat} & \textbf{Zutat} & \textbf{Maßeinheit} \\
		\hline
		Quark & 1.000 & g \\
		Zucker & 250 & g \\
		\hline
		Margarine & 125 & g \\
		feiner Grieß & 2 & Esslöffel \\
		\hline
		Kartoffelmehl & 1 & Esslöffel \\
		Backpulver & $\frac{1}{2}$ & Päckchen \\
		\hline
		Eier & 3 & Stück \\
		Zitronensaft & 1 & Esslöffel \\
		\hline
	\end{tabular}
\end{center}
\subsubsection{Zubereitung}
\begin{enumerate}[(1)]
	\item Man schlage die Eier auf und trenne Eigelb von Eiweiß. Man gebe dabei das Eiweiß in eine geeignete separate Schüssel (nicht aus Kunststoff), das Eigelb in eine große Schüssel.
	\item Man schlage das Eiweiß zu Eischnee. Optional kann man den Eischnee anschließend kaltstellen, bis er gebraucht wird.
	\item Man gebe Zucker und Margarine zum Eigelb in die große Schüssel hinzu.
	\item Man verrühre die Zutaten in der großen Schüssel gut.
	\item Man gebe feinen Grieß, Kartoffelmehl, Backpulver sowie Zitronensaft hinzu und verrühre erneut.
	\item Man gebe den Quark hinzu und verrühre bis eine homogene Masse entsteht.
	\item Man gebe den Eischnee hinzu und verrühre vorsichtig bei kleinster Stufe (unterheben).
	\item Man bereite eine runde Backform vor, gebe den Teig hinein und streiche ihn glatt.
\end{enumerate}
\subsubsection{Backzeit}
	Man backe den Kuchen bei Umluft und einer Temperatur von 180{\textcelsius} ca. 40 bis 45 Minuten lang.
%\subsubsection{Bilder}

\end{document}
