\documentclass[a4paper]{book}
\usepackage[textwidth=17cm, textheight=25cm]{geometry}

\input{../latex-std/lang-de.tex}

\usepackage{enumerate} % \begin{enumerate} ... \end{enumerate}
\usepackage{textcomp} % \textcelsius (°C) 

\title{Familienrezepte}
\date{Letzte Änderungen: \today}
\author{Maximilian Starke}

\begin{document}

\maketitle
\tableofcontents

\chapter*{Einleitung}

In diesem Buch sammeln wir Rezepte zum Kochen und Backen.
Jeder darf Rezepte einreichen. 

\chapter{Backrezepte}


\section{Rezepte mit Hefeteig}

\newpage
\subsection{Heidelbeer-Zimtstreusel-Kuchen}

\subsubsection{bereitgestellt von}
	Maximilian Starke
\subsubsection{Ursprung}
	Torgau in Sachsen (Deutschland)
\subsubsection{Zutaten}
\begin{center}
	\begin{tabular}{|l|rrl|r|}
		\hline
		\textbf{Zutat} & \textbf{Hefeteig} & \textbf{Streuselteig} & \textbf{Maßeinheit} & \textbf{Bemerkungen}\\
		\hline
		Mehl & 500 & 400 & g & \\
		Zucker & 150 & 200 & g & \\
		\hline
		Margarine & 200 & & g & + ein wenig für Backblech\\
		Butter & & 250 & g & \\
		\hline
		Salz & 1 & 1 & Prise & \\
		Milch & 125 & & ml & \\
		\hline
		Ei & 1 & & Stück & \\
		Hefe & 1 & & Stück & (42g) \\
		\hline
		Vanillinzucker & 1 & & Päckchen & (8g)\\
		Zimt & & 1 & gehäufte Teelöffel & (optional)\\
		\hline
		Heidelbeeren & 300 - 400 & & g & \\ % TODO Gewicht ermittteln beim nächsten Kuchen! ##
		(frisch oder gefroren) & & & & \\
		\hline
	\end{tabular}
\end{center}

\subsubsection{Zubereitung}
Alle Zutaten sollten nach Möglichkeit Zimmertemperatur angenommen haben. Falls jedoch gefrorene Heidelbeeren verwendet werden, sollten diese noch bis zum Zeitpunkt der Verwendung im Gefrierfach aufbewahrt werden.
\begin{enumerate}[(1)]
	\item Man bereite den Hefeteig vor:
	\begin{enumerate}[(\theenumi.1)]
		\item Man nehme eine Schüssel und siebe das Mehl dort hinein.
		\item Man gebe Margarine, Zucker, Salz, Ei und Vanillinzucker hinzu.
		\item Man erhitze die Milch auf ca. 40{\textcelsius} und verrühre die zerbröckelte Hefe mit einer Gabel in der Milch.
		\item Man gebe Milch und Hefe in die Schüssel.
		\item Man knete mit den Händen den Teig kräftig durch.
		\item Man stelle den Teig für 30 bis 90 Minuten zum Gehen an einen warmen Ort (40{\textcelsius} bis 60{\textcelsius}). Es empfiehlt sich beispielsweise, den Teig mit Schüssel in den Ofen zu stellen und diesen auf 50{\textcelsius} heizen zu lassen.
	\end{enumerate}
	\item Man bereite den Streuselteig vor:
	\begin{enumerate}[(\theenumi.1)]
		\item Man nehme eine Schüssel und siebe das Mehl dort hinein. 
		\item Man gebe Butter, Zucker, Salz, und Zimt hinzu.
		\item Man knete mit den Händen den Teig kräftig durch.
	\end{enumerate}
	\item Man fette ein Backblech mit Margarine ein.
	\item Man rolle den Hefeteig gleichmäßig auf dem Backblech aus.
	\item Man verteile die Heidelbeeren gleichmäßig oder nach gewünschtem Muster auf dem Hefeteig.
	\item Man streue den Streuselteig als Streusel in gewünschter Größe über den mit Heidelbeeren bedeckten Hefeteig.

\end{enumerate}
\subsubsection{Backzeit}
	Man backe den Kuchen bei Umluft und einer Temperatur von 175{\textcelsius} ca. 35 Minuten lang, bis die Streusel hellgelb sind. Wer möchte, kann den Ofen nach der halben Backzeit auf Oberhitze oder Oberhitze kombiniert mit Unterhitze umstellen, um eine kräftigere Bräunung der Streusel zu bewirken.




\section{Rezepte mit Quark}

\newpage
\subsection{Quark-Torte}

\subsubsection{bereitgestellt von}
	Maximilian Starke
\subsubsection{Ursprung}
	Torgau in Sachsen (Deutschland)
\subsubsection{Zutaten}
\begin{center}
	\begin{tabular}{|l|rl|}
		\hline
		\textbf{Zutat} & \textbf{Menge} & \textbf{Maßeinheit} \\
		\hline
		Quark & 1.000 & g \\
		Zucker & 250 & g \\
		\hline
		Margarine & 125 & g \\
		feiner Grieß & 2 & Esslöffel \\
		\hline
		Kartoffelmehl & 1 & Esslöffel \\
		Backpulver & $\frac{1}{2}$ & Päckchen \\
		\hline
		Eier & 3 & Stück \\
		Zitronensaft & 1 & Esslöffel \\
		\hline
	\end{tabular}
\end{center}
\subsubsection{Zubereitung}
\begin{enumerate}[(1)]
	\item Man schlage die Eier auf und trenne Eigelb von Eiweiß. Man gebe dabei das Eiweiß in eine geeignete separate Schüssel (nicht aus Kunststoff), das Eigelb in eine große Schüssel.
	\item Man schlage das Eiweiß zu Eischnee. Optional kann man den Eischnee anschließend kaltstellen, bis er gebraucht wird.
	\item Man gebe Zucker und Margarine zum Eigelb in die große Schüssel hinzu.
	\item Man verrühre die Zutaten in der großen Schüssel gut.
	\item Man gebe feinen Grieß, Kartoffelmehl, Backpulver sowie Zitronensaft hinzu und verrühre erneut.
	\item Man gebe den Quark hinzu und verrühre bis eine homogene Masse entsteht.
	\item Man gebe den Eischnee hinzu und verrühre vorsichtig bei kleinster Stufe (unterheben).
	\item Man bereite eine runde Backform vor, gebe den Teig hinein und streiche ihn glatt.
\end{enumerate}
\subsubsection{Backzeit}
	Man backe den Kuchen bei Umluft und einer Temperatur von 180{\textcelsius} ca. 40 bis 45 Minuten lang.
%\subsubsection{Bilder}



\newpage
\subsection{Quark-Torte}

\subsubsection{bereitgestellt von}
Elsa Goletz
\subsubsection{Ursprung}
NN in {Region} (Deutschland) %bitte ergänzen
\subsubsection{Zutaten}
\begin{center}
	\begin{tabular}{|l|rl|}
		\hline
		\textbf{Zutat} & \textbf{Menge} & \textbf{Maßeinheit} \\
		\hline
		Quark (Magerstufe) & 1.000 & g \\
		Zucker & 280 & g \\
		\hline
		Eier & 6 & Stück \\
		weiche Butter & 150 & g \\
		\hline
		feiner Grieß & 3 & Esslöffel \\
		Mehl & 1 & Esslöffel \\
		\hline
		Mandeln (gemahlen) & 100 & g \\ %frag nach ob iO
		Zitronen & 2 & Stück (je 222g) \\
		\hline
		Backpulver & $\frac{1}{2}$ & Päckchen \\ %frag nach ob iO
		Vanillezucker  & 1 & Päckchen \\
		\hline
		Salz & 1 & Prise  \\
		\hline
	\end{tabular}
\end{center}
\subsubsection{Zubereitung}
\begin{enumerate}[(1)]
	\item Man schlage die Eier auf und trenne Eigelb von Eiweiß. Man gebe dabei das Eiweiß in eine geeignete separate Schüssel (nicht aus Kunststoff), das Eigelb in eine große Schüssel.
	\item Man schlage das Eiweiß nach Zugabe einer Prise Salz zu Eischnee.
	\item Man gebe Zucker, Vanillezucker, die Schalte einer Zitrone (alternativ: 2 Teelöffel Zitronenraspel) und den Saft beider Zitronen zum Eigelb in die große Schüssel hinzu.
	\item Man verrühre die Zutaten in der großen Schüssel gut.
	\item Man gebe Quark, weiche Butter, Grieß und Mandeln hinzu.
	\item Man vermische Backpulver und Mehl und gebe dies in die große Schüssel hinzu.
	\item Man verrühre bis eine homogene Masse entsteht.
	\item Man hebe den steifen Eischnee unter.
	\item Man bereite eine runde Backform vor, gebe den Teig hinein und streiche ihn glatt.
\end{enumerate}
\subsubsection{Backzeit}

Im vorgeheizten Herd auf der zweiten Schiene bei 160 - 190 Grad ca. 80 - 90 Min. backen. (bei 80 min ohne Vorheizen verbrannt (Ober+Unterhitze), teste mit 55 Minuten? oder ohn Oberhitze)
Den Kuchen in der Form leicht auskühlen lassen, evtl. nach 15 Min. mit einem Messer den Formrand lösen, nach völligem Erkalten Formrand abnehmen, den Kuchen vom Formboden auf eine Tortenplatte schieben. Man kann nach Belieben nun auch Puderzucker aufstäuben.    

alter text:
Man backe den Kuchen bei Umluft und einer Temperatur von 180{\textcelsius} ca. 40 bis 45 Minuten lang.
%\subsubsection{Bilder}

\end{document}