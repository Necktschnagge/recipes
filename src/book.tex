\documentclass[a4paper]{book}
\usepackage[textwidth=17cm, textheight=25cm]{geometry}

\input{../latex-std/lang-de.tex}

\usepackage{enumerate}
\usepackage{textcomp}

\title{Rezeptsammlung von Informatikern}
\date{Letzte Änderungen: \today}
\author{Maximilian Starke}

\begin{document}

\maketitle
\tableofcontents

\chapter*{Einleitung}

In diesem Buch sammeln wir Rezepte zum Kochen und Backen.
Jeder darf Rezepte einreichen. 

\chapter{Backrezepte}

\section{Rezepte mit Hefeteig}

\newpage
\subsection{Heidelbeer-Zimtstreusel-Kuchen}

\subsubsection{bereitgestellt von}
	Maximilian Starke
\subsubsection{Ursprung}
	Torgau in Sachsen (Deutschland)
\subsubsection{Zutaten}
\begin{center}
	\begin{tabular}{|l|rrl|r|}
		\hline
		\textbf{Zutat} & \textbf{Hefeteig} & \textbf{Streuselteig} & \textbf{Maßeinheit} & \textbf{Bemerkungen}\\
		\hline
		Mehl & 500 & 400 & g & \\
		Zucker & 150 & 200 & g & \\
		\hline
		Margarine & 200 & & g & + ein wenig für Backblech\\
		Butter & & 250 & g & \\
		\hline
		Salz & 1 & 1 & Prise & \\
		Milch & 125 & & ml & \\
		\hline
		Ei & 1 & & Stück & \\
		Hefe & 1 & & Stück & (42g) \\
		\hline
		Vanillinzucker & 1 & & Päckchen & (8g)\\
		Zimt & & 2 & gehäufte Teelöffel & (optional)\\
		\hline
		Heidelbeeren & 200 & & g & \\ % TODO Gewicht ermittteln beim nächsten Kuchen! ##
		(frisch oder gefroren) & & & & \\
		\hline
	\end{tabular}
\end{center}
%\subsubsection{Geräte}
%\begin{itemize}
%	\item 2 Schüsseln (für Hefeteig und Streusel)
%	\item 1 feines Sieb (für Mehl)
%	\item 1 Kochplatte (für Milcherwärmung)
%	\item 1 Kochtopf (für Milcherwärmung)
%	\item 1 Gabel (für Hefezerkleinerung und Rühren)
%	\item 1 Backblech
%	\item 1 Nudelholz (kann eingespart werden, wenn man den Teig mit den Händen auf dem Blech verteilt.) % TODO heißt das wirklich so?
%\end{itemize}

\subsubsection{Zubereitung}
Alle Zutaten sollten nach Möglichkeit Zimmertemperatur angenommen haben. Falls jedoch gefrorene Heidelbeeren verwendet werden, sollten diese noch bis zum Zeitpunkt der Verwendung im Gefrierfach aufbewahrt werden.
\begin{enumerate}[(1)]
	\item Man bereite den Hefeteig vor:
	\begin{enumerate}[(\theenumi.1)]
		\item Man nehme eine Schüssel und siebe das Mehl dort hinein.
		\item Man gebe Margarine, Zucker, Salz, Ei und Vanillinzucker hinzu.
		\item Man erhitze die Milch auf ca. 40{\textcelsius} und verrühre die zerbröckelte Hefe mit einer Gabel in der Milch.
		\item Man gebe Milch und Hefe in die Schüssel.
		\item Man knete mit den Händen den Teig kräftig durch.
		\item Man stelle den Teig für 30 bis 90 Minuten zum Gehen an einen warmen Ort (40{\textcelsius} bis 60{\textcelsius}). Es empfiehlt sich beispielsweise, den Teig mit Schüssel in den Ofen zu stellen und diesen auf 50{\textcelsius} heizen zu lassen.
	\end{enumerate}
	\item Man bereite den Streuselteig vor:
	\begin{enumerate}[(\theenumi.1)]
		\item Man nehme eine Schüssel und siebe das Mehl dort hinein. 
		\item Man gebe Butter, Zucker, Salz, und Zimt hinzu.
		\item Man knete mit den Händen den Teig kräftig durch.
	\end{enumerate}
	\item Man fette ein Backblech mit Margarine ein.
	\item Man rolle den Hefeteig gleichmäßig auf dem Backblech aus.
	\item Man verteile die Heidelbeeren gleichmäßig oder nach gewünschtem Muster auf dem Hefeteig.
	\item Man streue den Streuselteig als Streusel in gewünschter Größe über den mit Heidelbeeren bedeckten Hefeteig.

\end{enumerate}
\subsubsection{Backzeit}
	Man backe den Kuchen bei Umluft und einer Temperatur von 175{\textcelsius} ca. 35 Minuten lang, bis die Streusel hellgelb sind. Wer möchte, kann den Ofen nach der halben Backzeit auf Oberhitze oder Oberhitze kombiniert mit Unterhitze umstellen, um eine kräftigere Bräunung der Streusel zu bewirken.
\end{document}